\documentclass[final,10pt,reqno,twoside]{article}

% Harsha Lokavarapu
% Department of Mathematics
% University of California, Davis
%
%
% Copyright:
%
%  Tuesday March 27, 2017 at 1801 PDT
%
% Created:
%
%  Tuesday March 27, 2017 at 1801 PDT
%
% Revision History:
%
%  Revision 0.00
%  Revision 0.01: March 28th, 2017 at 1200 PDT
%
%%%%%%%%%%%%%%%%%%%%%%%%%%%%%%%%%%%%%%%%% LOAD AMS LaTeX Packages %%%%%%%%%%%%%%%%%%%%%%%%%%%%%%%%%%%%%%%%%

% Using the "amsart" or "amsbook" document class instead of the standard LaTeX "article" document class
% provides all (or at least most) of the AMS math macros. However, it also puts a colon in the description
% list environment, which I don't like.

\usepackage{amsthm}
\usepackage{amsmath}
\usepackage{amssymb}
\usepackage{amsfonts}

% Added by EGP for script fonts

\usepackage[mathscr]{euscript}

% Added by EGP for script L for a linear operator: \mathscr{L}

\usepackage{mathrsfs}

% This package provides an arrow, \longrightarrow, that is longer than AMSMATH's \xrightarrow and
% \xrightarrow.

\usepackage{extarrows} % (extensible arrows)

%%%%%%%%%%%%%%%%%%%Additional packages%%%%%%%%%%%%%%%%5
\usepackage{url}

%The goto Geochemist package.
\usepackage[version=4]{mhchem}
%%%%%%%%%%%%%%%%%%%%%%%%%%%%%%%%%%%%%%%%% SET UP PAGE PARAMETERS %%%%%%%%%%%%%%%%%%%%%%%%%%%%%%%%%%%%%%%%%%

% EGP's preferred page style for notes, homework assignments, exams, etc.

% Margins, paragraph indents, space between paragraphs if any, etc. Good references include page 85 of
% "The LaTeX Companion" by Frank Mittelbach and Michael Goossens and page 260 of "Math Into LaTeX" by
% George Gratzer.

% Enlarge the width and height of the printed page

\setlength{\textwidth }{7.50 in}
\setlength{\textheight}{9.25 in}

% Space between the end of the odd and even side margins and the beginning of the text.

\setlength{\oddsidemargin }{0.00 in}
\setlength{\evensidemargin}{0.00 in}

% The side margins are 1.0 inch plus \hoffset and the top margins is 1.0 inch plus \voffset.
% According to "The LaTeX Companion" the default values are \hoffset = 00 pt and \voffset = 00 pt.

\setlength{\hoffset}{-0.50 in}

% This is the width of the top margin plus one inch

\setlength{\voffset}{-0.50 in}

% If there is no header in this document, set each of the following values to zero. According to "Math
% Into LaTeX"  by George Gratzer the default values for LaTeX's article documents are \headsep = 25pt,
% \headheight = 12pt, and  \topmargin = 16pt.

\setlength{\headsep   }{12pt}

% Package Fancyhdr Warning: \headheight is too small (12.0pt): Make it at least 14.0pt.

\setlength{\headheight}{14pt}

\setlength{\topmargin }{00pt}

% This is the amount of space between the last line on the page and the footer. According to "The LaTeX
% Companion" and "Math Into LaTeX" the default value is \footskip = 30 pt.

\setlength{\footskip}{0.50 in}

% I like space between paragraphs, since it makes the document more readable. However, this does not seem
% to change the spacing between paragraphs contained in an item of a list.

\setlength{\parskip}{12pt}

% Comment out the following line to use the default amount to indent the first line of each paragraph.

\setlength{\parindent}{00pt}

% \fboxsep is the distance between the edge of a box and the text the box contains.

\setlength{\fboxsep}{10pt}

% There is no need for BibTeX in this exam so I turned it off, since it creates unwanted files.

\let\nobibtex = t

%%%%%%%%%%%%%%%%%%%%%%%%%%%%%%%%%%%%%%%% Load The GRAPHICX Package %%%%%%%%%%%%%%%%%%%%%%%%%%%%%%%%%%%%%%%%

% The best reference I have found for including graphics in figures and especially manipulating captions
% and subcaptions is "The LaTeX Companion" by Frank Mittelbach and Michel Goossens. The following packages
% are described there.

% WARNING: PDFTeX does not recognize eps files. They MUST be converted to PDF files.

\usepackage{graphics}

\DeclareGraphicsExtensions{.png}

% This tells LaTeX to look in the subdirectory ./Figures for my PDF files. Note that the folder name must
% be in braces {./Figures}, even if there is only one folder in the list.

\graphicspath{{./00Figures/}}

%%%%%%%%%%%%%%%%%%%%%%%%%%%%%%%%%%%%%%%%% Set Up The CHANGES Package %%%%%%%%%%%%%%%%%%%%%%%%%%%%%%%%%%%%%%%%%

\usepackage[draft]{changes}

% Anonymous: To use this feature type \added[]{ ... }

% Footnotes: To put an \added[EGP]{...} comment in a footnote use \added[-]{\footnote{\added[EGP]{ ... }}}.
% For example, \added[*]{\footnote{\added[EGP]{Double check this equation.}}}

\definechangesauthor[name={Footnotes}, color=red]{*}

\setauthormarkupposition{left}

% To use this feature type \added[Delete]{ ... }
% Note: Version 1.0.0 can special strike out deleted text. See the manual.

\definechangesauthor[color=green]{Delete}

% To use this feature type \added[id=EGP]{ ... } !!!!

\definechangesauthor[name={Elbridge Gerry Puckett}, color=red]{EGP}

% Test from http://tex.stackexchange.com/questions/65453/track-changes-in-latex/65466#65466, which worked on
% Friday, January 17, 2014 at 10:58 PST

% \definechangesauthor[name={Per cusse}, color=orange]{per}

% This is \added[id=per,remark={we need this}]{new} text.
% This is \added[id=per,remark={has to be in it}]{new} text.
% This is \deleted[id=per,remark=obsolete]{unnecessary}text.
% This is \replaced[id=per]{nice}{bad} text.

%%%%%%%%%%%%%%%%%%%%%%%%%%%%%%%%%%%%%%%% Set Up The CHANGES Package %%%%%%%%%%%%%%%%%%%%%%%%%%%%%%%%%%%%%%%%

% \usepackage[draft]{changes}

% Anonymous: To use this feature type \added[]{ ... }

% Footnotes: To put an \added[EGP]{...} comment in a footnote use \added[-]{\footnote{\added[EGP]{ ... }}}.
% For example, \added[*]{\footnote{\added[EGP]{Double check this equation.}}}

% \definechangesauthor[name={Footnotes}, color=red]{*}

% \setauthormarkupposition{left}

% To use this feature type \added[Delete]{ ... }
% Note: Version 1.0.0 can special strike out deleted text. See the manual.

% \definechangesauthor[color=green]{Delete}

% To use this feature type \added[EGP]{ ... }

% \definechangesauthor[name={Elbridge Gerry Puckett}, color=red]{-EGP}

%%%%%%%%%%%%%%%%%%%%%%%%%%%%%%%%%%%%%%% LOAD THE "enumitem" PACKAGE %%%%%%%%%%%%%%%%%%%%%%%%%%%%%%%%%%%%%%%

\usepackage{enumitem}

% \setenumerate[1]{labelindent=0pt,itemindent=12pt}

\setlength{\labelwidth}{00 pt}
\setlength{\leftmargin}{0.0in}

%%%%%%%%%%%%%%%%%%%%%%%%%%%%%%%%%%%%%%% Set Up The "Version" Package %%%%%%%%%%%%%%%%%%%%%%%%%%%%%%%%%%%%%%%

% This environment allows one to include text in the LaTeX file that is not printed when it is compiled
% CAUTION: Do not use "space" as the name of a version environment, since it causes problems with LaTeX's
% memory

\usepackage{version}

\includeversion{Solutions}
\excludeversion{No_Solutions}

% Do or do not include the problem of showing the column space of A is a subspace of R^m

\includeversion{Problem_01(g)}

\includeversion{Problem_06}

\includeversion{Problem_07_Solutions}
\excludeversion{Problem_07_No_Solutions}

% These are version environments I use while creating the exam in order to keep track of the parts that are
% finished and those parts that need work.

% STILL ACTIVE ON SATURDAY, FEBRUARY 27, 2016

\excludeversion{red}
\excludeversion{hidered}

\excludeversion{finished}
\excludeversion{notfinished}

% If one wants to edit the solution to a particular problem while having the other problems print with or
% without Solutions, then use these version for the problem being edited.

%\excludeversion{Solutions}
%\includeversion{No_Solutions}

%%%%%%%%%%%%%%%%%%%%%%%%%%%%%%%%%%%%%% Set Up Running Head and Foot %%%%%%%%%%%%%%%%%%%%%%%%%%%%%%%%%%%%%%

% LaTeX macro package to print the date and the time of day.

% There are various declarations that change the effect of \today. The change can be localized by placing
% the declaration within a group.
%
%  <Day> <Month> <Year> formats:
%
%   \longdate   The declaration "\longdate" will redefine \today to produce the current date displayed in
%               the form
%
%               Wednesday 8th March, 2000
%
% if the package option dayofweek is used (Default), or
%
%               8th March, 2000
%
% if the package option nodayofweek is used.
%
%   \shortdate  The declaration "\shortdate" will redefine \today to produce the current date displayed in
%               the form Wed 8th Mar, 2000 if the package option dayofweek is used, or
%
%               8th Mar, 2000
%
% if the package option nodayofweek is used.

\usepackage[short]{datetime}

% The current time is displayed using the command \currenttime. The format can be changed using the
% declaration \settimeformat{<style>}, where <style> is the name of the format. Available formats are:
%
%   xxivtime     Twenty-four hour time in the form 22:28 (Default)
%
%   ampmtime     Twelve hour time in the form 10:28pm
%
%   oclock       Displays the current time as a string, e.g. Twenty-Eight minutes past Ten in the afternoon.

\settimeformat{xxivtime}

% This is the day that I wrote the last version of this exam, assuming I remembered to update this date.

\newdate{Date-Written}{02}{03}{2016}

% "fancyhdr" is an easy to use package for making headers and footers.

\usepackage{fancyhdr}

\pagestyle{fancy}

% Clear all header and footer fields.

\fancyhead{}

%%%%%%%%%%%%%%%%%%%%%%%%%%%%%%%%%%%%%%%%%%%%%%%%%%%%%%%%%%%%%%%%%%%%%%%%%%%%%%%%%%%%%%%%%%%%%%%%%%%%%%%%%%%%%%%
% "L" stands for "Left", "C" for "Centered", "R" for "Right", "O" for Odd page and "E" for Even page

\fancyhead[CO]{\textbf{Geology Journal}}
\fancyhead[CE]{\textbf{Geology Journal}}


% FOOTER

\fancyfoot[LO]{\copyright~Harsha~Lokavarapu~2017~~~~\emph{Revision~0.01}}
\fancyfoot[LE]{\copyright~Harsha~Lokavarapu~2017~~~~\emph{Revision~0.01}}

\fancyfoot[CO]{--~\thepage~--}
\fancyfoot[CE]{--~\thepage~--}

\fancyfoot[RO]{\today~at~\currenttime}
\fancyfoot[RE]{\today~at~\currenttime}

%%%%%%%%%%%%%%%%%%%%%%%%%%%%%%%%%%%%%%%%% LOAD EGP's LaTeX MACROS %%%%%%%%%%%%%%%%%%%%%%%%%%%%%%%%%%%%%%%%%

% EGP's Macro definitions for notes

\include{EGPs_Note_Macros}

% EGP's Macro definitions for Math 22A

% \include{EGPs_Linear_Algebra_Macros}

% I want the cover page to be page number 0.

% \setcounter{page}{0}

%%%%%%%%%%%%%%%%%%%%%%%%%%%%%%%%%%%%%%%% LOAD THE HYPERREF PACKAGE %%%%%%%%%%%%%%%%%%%%%%%%%%%%%%%%%%%%%%%%

% The basic usage with the standard settings is straightforward. Just load the package in the preamble, at
% the end of all the other packages but prior to other settings.

% From the `Introduction' to the hyperref manual: Make sure it comes last of your loaded packages, to give
% it a fighting chance of not being over-written, since its job is to redefine many LATEX commands.
% Hopefully you will find that all cross-references work correctly as hypertext. For example, \section
% commands will produce a bookmark and a link, whereas \section* commands will only show links when paired
% with a corresponding \addcontentsline command.

\usepackage[bookmarksopen=false]{hyperref}

% Set the internal hyperref links, such as equation numbers, to black for photocopying the exam. Wikibooks
% has an article on color names at http://en.wikibooks.org/wiki/LaTeX/Colors

\hypersetup{linkcolor=blue}

% bookmarksopen=true means only level 1. % are displayed.% Setup hyperref parameters. See the hyperref manual
% for details.

\hypersetup{colorlinks=true, urlcolor=blue, pdftitle={2016 REU LaTeXSample}}

% The default values for "bookmarks" is 'false' and for "bookmarksopen" is 'false'

\hypersetup{pdfauthor={\textcopyright\ Harsha Lokavarapu}}

% Uncomment if you want the first page to be page number 0.

% \setcounter{page}{0}

\begin{document}


%%%%%%%%%%%%%%%%%%%%%%%%%%%% NO LINE BELOW THE HEADER EXCEPT ON THE COVER PAGE %%%%%%%%%%%%%%%%%%%%%%%%%%%%

% \renewcommand{\headrulewidth}{0.0pt}


% BEGINNING OF THE PROBLEMS ON THE MATH 22A SECTION 002 PRACTICE FINAL EXAM FOR WINTER QUARTER 2013

% I wanted four levels of bookmarks beyond level zero, which apparently is not possible. Therefore, I
% changed the level 01 bookmarks to level 00, etc.

% \pdfbookmark[0]{Problems}{Problems}

% This cannot appear before the first "\item" inside the first list environment "\begin{enumerate}"

\begin{finished}
  \begin{center}
    \added[id=EGP]{PROBLEMS ONE (a)-(d) AND THE SOLUTIONS TO PROBLEM ONE ARE FINISHED.}
    \added[id=EGP]{PLEASE CHECK THEM!}
  \end{center}
\end{finished}

% Set the labels in the first level of the enumerated list below to be \textbf{Problem~0\arabic{enumi}}
% See page 347 of "Math Into LaTeX".

\renewcommand{\theenumi}{\textbf{ITEM~0\arabic{enumi}}}
\renewcommand{\labelenumi}{\theenumi}

%                                  PROBLEM ONE ON PAGE ONE
%
%--------------------------------------------------------------------------------------------------------------
%gi
% PAGE 01 COMPUTATION 01
% Set the horizontal spacing for the first level of enumerated lists. The key "align=left" is essential to
% not having the first level of indentation be as large as the level one item labels, which are "Problem~01",
% etc.

  \begin{enumerate}[leftmargin=0cm, align=left]

    \pdfbookmark[0]{Computation~01}{Computation_01}

    % PROBLEMS 01 (a) and (b) ON PAGE ONE

    % SUBSPACES



    \pdfbookmark[1]{Item_01}{Item_01}
    \item \label{Prob:Item_01} Understand how to convert gigatons of carbon to parts per million in the atmosphere.
             \begin{enumerate}[label=\textbf{(\alph*})]
   				 \item \path{https://micpohling.wordpress.com/2007/03/30/math-how-much-co2-by-weight-in-the-atmosphere/}
   				 \item Parts per million by volume and parts per million by mass. Conversion ratio of molar mass of C by molar mass of CO$_2$.
   				 \item Note that the ppm of carbon emissions is changing annually. Currently, the value is 400 ppm.
             \end{enumerate}
    \pdfbookmark[1]{Item_02}{Item_02}
    \item Read the manuscript "Ingassing, Storage, and Outgassing of Terrestrial Carbon through Geologic Time" by Rajdeep Dasgupta.

    \pdfbookmark[1]{Item_03}{Item_03}
    \item Research the carbon fluctuations in the atmosphere during the phenozoric period. In particular analyze the two plots at \path{https://en.wikipedia.org/wiki/Carbon_dioxide_in_Earth\%27s_atmosphere}.
         \begin{enumerate}[label=\textbf{(\alph*})]
           \item Goals: 
             \begin{enumerate}
               \item Understand how the measurements are being made in terms of temperature, $CO_2$, etc.
             \end{enumerate}
           \item \includegraphics{00FIGURES/Phanerozoic_Carbon_Dioxide.png}
           \item \includegraphics{00FIGURES/Carbon_Dioxide_400kyr.png}
         \end{enumerate} 
         
    \pdfbookmark[1]{Item_04}{Item_04}
    \item Investigate plot at \path{https://en.wikipedia.org/wiki/Geologic_temperature_record}
         \begin{enumerate}[label=\textbf{(\alph*})]
           \item 
              \includegraphics{00FIGURES/Phanerozoic_Climate_Change.png}
         \end{enumerate}
         
    \pdfbookmark[1]{Item_05}{Item_05}
    \item Read the paper \path{00PAPERS/Atmospheric_CO2_and_O2_During_Phanerozoic_Tools_Patterns.pdf}. If possible, print a hard copy of this paper for Don such that the chemical equations aren't quiet pigeon goo. 
         \begin{enumerate}[label=\textbf{(\alph*})]
           \item Status: Start reading up to 6.11.2.2 GEOCARB Models as of 1156 PDT 3/28/17
           \item Questions:
          	\begin{enumerate} 
          		\item What, who and how did GEOCARBB model come to be?         
          		\item Are \ce{CaSiO3} and \ce{MgSiO3} metamorphic rocks?
          		\item the weathering of \ce{Ca^{2+}} and \ce{Mg^{2+}} silicate rocks consumes a stoichiometrically equivalent amount of \ce{CO2}. Stoichiometrically equivalent??
         	\end{enumerate}
         	\item Notes:
         	\begin{enumerate}
         		\item Carbonic acid is chemically written as \ce{HCO3^-}. 
         		\item \ce{Ca^{2+}}, \ce{Mg^{2+}} ions are derived from \ce{CaSiO3} and \ce{MgSiO3} reacting with \ce{HCO2^-}.
         		\item Weathering of calcium silicates, one major \ce{CO2} sink is chemically written as \\
         		   \begin{eqnarray}
         		        \ce{2CO2 + H2O + CaSiO3 -> Ca^{2+} + SiO2 + 2HCO3^-} \\
         		        \ce{Ca^{2+} + 2HCO3^- -> CaCO3 + CO2 + H2O} 
         		   \end{eqnarray}
         		   
         		   and adding the two equations gives us
         		   \begin{equation}
         		   		\ce{CO2 + CaSiO3 -> CaCO3 + SiO2}
         		   \end{equation}
         		   Notice that, Carbonate formation releases a stoichiometrically equivalent amount of \ce{CO2} , but the weathering of one mole of silicate mineral consumes two moles of \ce{HCO3^-}. The weathering of \ce{Ca^{2+}} and \ce{Mg^{2+}} silicate rocks consumes a stoichiometrically equivalent amount of \ce{CO2}.
         		   
         		   In this Treatise, the identification of these processes are attributed to French chemist and mining engineer J.J. Ebelmen in 1845. Urey is credited with a more modern treatment in 1952.
         		\item The second major \ce{CO2} sink is the burial of organic material on land (or in the ocean). Physcial seperation of buried \ce{CH2O} is released ten to 100 million years as a result of plate tectonic forces either through degassing, or chemical weathering adding \ce{C} into the atmosphere. Coined as geo-respiration. The inverse process of burial is coined as geo-photosynthesis. The chemical equation is written as
         		    \begin{equation}
         		    	\ce{CO2 + H2O -> CH2O + O2}
         		    \end{equation}
         		\item \ce{CO2} and \ce{O2} are decoupled because the processes that influence these quantities in the atmosphere are independent of each other.
         		\item The short term carbon cycle influences the atmosphere on time scales of thousands of years. The components that influence the short term carbon cycle can be decoupled for the components that influence the long term carbon cycle. Short term carbon cycle determines the fluxes of carbon in the surface reservoirs.  
         		\item Long term carbon cycle is influenced by its constitutive bells and whistles on the time scale of > $10^5$ years.   
         	\end{enumerate}
         \end{enumerate}    
         \pdfbookmark[1]{Item_06}{Item_06}
         \item Overlay of CO$_2$ variation and temperature data
            \begin{enumerate}[label=\textbf{(\alph*})]
            	\item PAPER: Climate and atmospheric history of the past 420,000 years from the Vostok ice core, Antarctica
            	\item Data set: ftp://ftp.ncdc.noaa.gov/pub/data/paleo/icecore/antarctica/vostok/
            	\item Created using inkscape \\
            	   \includegraphics{00FIGURES/Temperature_variation_and_CO2.png}
            	\item Using gnuplot, generated this superimposed plot with custom axis and/or colors with data available at above repository.
            	   \includegraphics{00DATA/Vostok_Ice_Core/VOSTOK_T_CO2.png}
            \end{enumerate}
              
 \end{enumerate}

  % THE BIBLIOGRAPHY STYLE AND THE BiBTeX FILE COME AT THE END OF THE LaTeX FILE.
  
%  \bibliographystyle{plain}
%  \bibliography{Notebook}

\end{document} 
