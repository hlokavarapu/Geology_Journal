A second set of corrections of \textbf{A Reservoir Model for the Evolution of Carbon in the Earth's Mantle} is on my desk. I picked up the corrections and completed them in a record time of a half hour to hour max. My eyes were not glued to the ticking clock unfortunately. Don, handed my two problems that are equivalently the start of the reservoir model. 

Problem 1

Atmosphere loses carbon to the continental crust

\begin{equation}
  \frac{d^{c}M_{a}}{dt} = -^{c}F_{a-cc}
\end{equation}

\begin{equation}
  ^{c}F_{a-cc} = 0 \quad at \quad 0 < t < t_0
\end{equation}

\begin{equation}
  ^{c}M_{a} = \text{mass of carbon in atmosphere}
\end{equation}

\begin{equation}
  ^{c}F_{a-cc} = \text{flux of carbon from atmosphere to continental crust}
\end{equation}

\begin{equation}
  ^{c}F_{a-cc} = \frac{^{c}M_{a}}{^{c}M_{a0}} ^{c}F_{(a-cc)t_{0}}
\end{equation}

\begin{equation}
  ^{c}M_{a0} = \text{mass of carbon in the atmosphere at} \quad t = 0 - t_{0}
\end{equation}

\begin{equation}
  ^{c}F_{(a-cc)t_{0}} = \text{flux of carbon from atmosphere to continental crust at} \quad t=t_0
\end{equation}

\begin{equation}
  ^{c}F_{(a-cc)t_{0}} = \frac{^{c}M_{a0}}{\tau_{a-cc}}
\end{equation}

\begin{equation}
  \tau_{a-cc} \text{is a specified time constant}
\end{equation}

\begin{equation}
  \frac{d^{c}M_{a}}{dt} = - \frac{^{c}M_a}{\tau_{a-cc}}
\end{equation}

\begin{equation}
  \text{Initial condition} \quad ^{c}M_{a} = ^{c}M_{a0} \quad \text{at} \quad t=t_{0}
\end{equation}

Determine $^{c}M_{a}$ for time period $t=0$ to $t=4.4$ Gyr. Require $^{c}M_{a} = ^{c}M_{a0} = 1.57 \times 10^8$ Gt at $t = t_0$. Take $\tau_{a-cc} = 1$ Gyr, $t_{0} = 0.5$ Gyr.

1. Obtain analytial solution.
2. Obtain numerical solution. (Hint suggested integration alogrithm is newton's method.)
